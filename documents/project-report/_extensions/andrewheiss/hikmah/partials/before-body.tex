
% ---------------
% TITLE SECTION
% ---------------
$if(published)$
$if(code-repo)$
\published{\textbf{$date$} \qquad $published$ \\ {\scriptsize $code-repo$}}
$else$
\published{\textbf{$date$} \qquad {\scriptsize $published$}}
$endif$
$else$
$if(code-repo)$
\published{\textbf{$date$} \\ {\scriptsize $code-repo$}}
$else$
\published{\textbf{$date$}}
$endif$
$endif$

$if(title)$
\maketitle
$endif$

\pagebreak
\begin{abstract} 
\\ \\
In spring of 2023 we reached out to Event Safety A/S regarding the possibility of a collaboration with them regarding our final project of our education: B.E in software technology.\\
The idea was a software application implementing artificial intelligence and computer vision to help improve crowd safety management.
Following an agreement to collaborate and various festival invitations by Event Safety to collect data,
a problem statement was produced: \textit{Can computer vision software and AI techniques be leveraged to improve crowd overview for security guards,
by receiving video feed from large crowds, and ultimately improve crowd safety?}
\\ \\
The projects structure and planning followed a waterfall model for the purposes of development.
Along with this, we had predefined meetings arranged with our collaborators from Event Safety to showcase progress and gather feedback for further development.
When supervision was required, we turned to our supervisor and scheduled a meeting, usually within a couple of days to quickly move on.
\\ \\
Research was conducted into the academic litterature on computer vision and existing crowd counting methods, leading to the usage of the SASNet crowd counting deep learning model.
After implementing the model successfully, the focus of the project shifted to image manipulation for better results. This includes, but is not limited to,
perspective correction, image upsampling and image downsampling. Backend implementation of the model was run on UCloud - a cloud computing service hosted by SDU - and a frontend
developed in Python for application of heatmaps and the ability to segment the crowd based on user drawn polygons.
When development of the project had concluded, validation and verification commensed. Manual count comparisons, and thourough user test through a focus group was held for employees at Event Safety to gather final feedback and perspectives on the project and its viability.

Based on the presentation and feedback, we are able to conclude that we succeeded in implementing computer vision and AI techniques in a crowd management system. Comparing the feedback from our collaborators to our initial requirements, 
the system has adequate precision for situations that makes it useable to the inteded user while allowing for development of more functionality.
The system can successfully be leveraged by safety experts to improve crowd safety and crowd safety management

\end{abstract}
\vskip 3em

$if(keywords)$
\begin{keywords}
\def\sep{;\ }
$for(keywords/allbutlast)$$keywords$\sep $endfor$
$for(keywords/last)$$keywords$$endfor$
\end{keywords}
$endif$

$if(epigraph)$
$for(epigraph)$
$if(epigraph.source)$
\epigraph{$epigraph.text$}{---$epigraph.source$}
$else$
\epigraph{$epigraph.text$}
$endif$
$endfor$
$endif$

\newpage 

% -------------------
% END TITLE SECTION
% -------------------


