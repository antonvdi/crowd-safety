
% ---------------
% TITLE SECTION
% ---------------
$if(published)$
$if(code-repo)$
\published{\textbf{$date$} \qquad $published$ \\ {\scriptsize $code-repo$}}
$else$
\published{\textbf{$date$} \qquad {\scriptsize $published$}}
$endif$
$else$
$if(code-repo)$
\published{\textbf{$date$} \\ {\scriptsize $code-repo$}}
$else$
\published{\textbf{$date$}}
$endif$
$endif$

$if(title)$
\maketitle
$endif$

\pagebreak
\begin{abstract} 
\\ \\
In spring 2023 we reached out to EventSafety A/S regarding the possibility of a collaboration with them regarding our final project.\\
The idea was a crowd management system using Aritifical intelligence and computer vision to help improve crowd safety management.
Following an agreement to collaborate and various festival invitations by EventSafety to collect data,
a problem statement was produced: \textit{Can computer vision software and AI techniques be leveraged to improve crowd overview for security guards,
by receiving video feed from large crowds, and ultimately improve crowd safety?}
\\ \\
The projects structure and planning followed a waterfall model for the purposes of development.
Along with this, we had predefined meetings arranged with our collaborators from EventSafety to showcase progress and gather feedback for further development.
When we needed supervision for our project we turned to our supervisor and scheduled a meeting, usually within a couple of days to quickly move on.
\\ \\
Research was conducted into existing crowd counting methods and their papers leading to the usage of the SASNet crowd counting deep learning model.
After implementing the model successfully, the focus of the project shifted to image manipulation for better results. This includes, but is not limited to,
perspective correction, distortion correction, image upsampling \& downsampling. Backend implementation of the model was run on UCloud - a cloud computing service hosted by SDU - and a frontend
developed in python for application of heatmaps and the ability to segment the crowd based on user drawn polygons.
When development of the project had completed, manual count comparisons, a thourough user test and questionnaire and presentation was held for other employees at Event Safety to gather final feedback and perspectives on the project and its usability.
Based on the presentation and feedback, we are able to conclude that we succeeded in implementing computer vision and AI techiques in a crowd management system. The system has precision that makes it useable to the inteded customer while allowing for further development of more functionality.
The system can successfully be leveraged to leverage and improve crowd safety and crowd safety management

\end{abstract}
\vskip 3em

$if(keywords)$
\begin{keywords}
\def\sep{;\ }
$for(keywords/allbutlast)$$keywords$\sep $endfor$
$for(keywords/last)$$keywords$$endfor$
\end{keywords}
$endif$

$if(epigraph)$
$for(epigraph)$
$if(epigraph.source)$
\epigraph{$epigraph.text$}{---$epigraph.source$}
$else$
\epigraph{$epigraph.text$}
$endif$
$endfor$
$endif$

\newpage 

% -------------------
% END TITLE SECTION
% -------------------


